INTRODUÇÃO:
O seu estudo,
faz-se dentro dele próprio,
ou seja
usamos Open Access para estudar o Open Access
ele passa-dentro de si,
Castells 2000 reparou que a revolução do milénio era precisamente caracterizada pela passagem da informação 
ao centro
a informação passa a ser
tanto feramenta quanto produto.
A informação passa a ter uma função
muito mais central,
a informação passa a ser
a verdadeira matéria prima.
O material em bruto.

Castells procura evidenciar a passagem (*ou mesmo desta evolução)  de uma máquina industrial global baseada em recursos minerais de extração para uma indústria de ponta mais evoluida que não só utiliza tecnologias digitais para se gerir, mas que também utiliza essas mesams tecnologias para criar o seu surpluss, o seu lucro, o seu valor e produto.

A sua forma de extração é brilhantemente estudada por Sasken. 
Sasken ao estudar os desaires da banca ao longo do século XX pode testemunhar bem de perto os primórdios
da revolução digital estudada e teorizada por castells.
e dos impactos da abertura ao mundo digital no efeito da globalizçao.

Assim a sociologa estuou
Assim omo o processo de extração de bit coin se chama mining. Tudo isto demonstra uma parefernália culural 


Assim como é apresentada pela socióloga a extração 
A sua extração é feita 

geri-la e usá-la passa a criar o seu valor.

o mundo digital é 
o solo de onde se extrai o valor
mais precisamente,
a informação está 

ANÁLISE:

Entrevistámos através de inquérito um conjunto de detentores de email da FLUP. Obtivemos 106 entradas, das quais removemos 17 por não apresentarem informação. Possivelmente o respondentes arrependeu-se, ou a ligação ao servidor caiu. Não temos forma de saber se a mesma pessoa respondeu mais do que uma vez ao inquérito, i.e., não era necessária identificação através de, por exemplo, endereço de email. Escolhemos que assim fosse para não reduzir o número de participantes, já de si reduzido. 

O inquérito foi distribuido potencialmente para 46598 detentores de endereço de email da FLUP e obtivemos participação máxima de 89 respondentes por pergunta.
A nossa população apresentou-se bastante jovem, com cerca de metade da fatia entre os 18 e os 24 anos de idade (55%). Ao todo, 72% tinham idade abaixo dos trinta, 14% estavam entre os 30 e os 40 anos e acima disso, 14%, representados estes ultimos por 13 pessoas apenas.
A representação na área das Humanidades, Artes e Ciencias Sociais foi muito superior à representação na área das Ciencias e Tecnologias. Assim 87% da nossa população afirmou estudar na área das humanidades, por oposição aos 13% que afirmaram  ser provenientes das áreas científicas.
Relativamente à distribuição na hierarquia académica, constatámos que 88% dos respondentes não tinham grau de doutor. 27% estavam a frequentar licenciatura, enquanto outros 27% já a tinham completado, enquanto que 32% tinham grau de mestre. 
Por um lado isto mostra que a amostra recolhida para este estudo é constituida sobretudo por estudantes e não tanto por investigadores, o que é expectável. No entanto é de notar que, segundo os dados, existe uma representação maior de mestres do que alunos de licenciatura ou já licenciados.  (MAPA DE DISPERSAO OU MULTICOLUNAS DA REPRESENTACAO DE IDADE/NIVEL ACADÉMICO)

De uma foram geral, a nossa população apresentou os seguintes resultados ao nosso inquérito:


Dada a distribuição da nossa população decidimos dividir o universo em tres fatias: uma das fatias, a maior, com toda a gente dos 18 aos 29. Outra fatia dos 30 aos 39. E outra fatia dos acima de 40.



%SCHEMA%

INTRO:
	TEORIA SOCIAL
	OPEN ACCESS
	CASTELLS
	SASSEN
METOOLOGIA:
ANALISE
TRABALHO FUTURO
CONCLUSAO
